4. Patrons de disseny utilitzats i la seva justificació

Pattern Adapter

Hem utilitzat el patró adaptador en la crida al servei extern de missatgeria. El motiu pel qual l’usem és per no crear un acoblament entre les classes del domini i el mateix servei. És a dir, en cas de modificació sobre la crida de l'API, únicament és veuran afectats els canvis sobre la classe adaptador, per tant no quedaran afectades la resta de classes que utilitzen aquest servei.
A més, a través de l’ús d’un adaptador podem modificar i/o filtrar les dades obtingudes d’una crida, definida pel servei extern, per tal d’adequar-les a les nostres preferències.

Pattern Strategy

Hem utilitzat el patró estràtegia, ja que existeixen dues modalitats diferents de puntuació. Aquestes dues classes únicament difereixen en el seu comportament. El tipus d’estratègia de puntuació s’aplica en funció del nombre de partides guanyades per part de l'usuari. A més utilitzant aquest tipus de patró ens assegurem que de cara a un futur serà possible afegir noves estrategies de puntuació amb les mínimes modificacions sobre l’esquema i el codi possibles.

Pattern Singleton

Hem utilitzat el patró singleton, ja que hi ha classes que únicament presenten una sola instància, a més en les classes a on s’aplica aquest patró únicament és crearan els objectes si prèviament no han estat creats, altrament si ja han estat creats, no els crea de nou.
En el nostre esquema s’ha aplicat aquest patró en les següents classes: Paràmetres, SeviceLocator, Factoria i CtrlDataFactoria. I aquestes han de ser accessibles per la resta de classes des d’un punt d’accés conegut.

Pattern Factory

Hem utilitzat el patró factoria, ja que així s’allibera a les classes creades d’aquelles responsabilitats que no li corresponen però que són necessàries per a la seva creació, mantenint la seva independència i no creant acoblament.
És a dir, l’utilitzem en aquells casos en què la creació d’un objecte implica quelcom més que una simple instanciació, per exemple, en el cas de la creació d’un controlador ens referim a l’accés que s’ha de realitzar a la capa de dades.

Access to the Data Layer

Hem utilitzat accessos a la capa de dades des dels controladors per tal de mantenir una bona cohesió entre les classes i no augmentar l’acoblament entre aquestes. Així únicament són els controladors els encarregats de realitzar aquests accessos mentre que les classes desconeixen l’existència de la resta de tipus.
